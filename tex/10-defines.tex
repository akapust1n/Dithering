\Defines % Необходимые определения. Вряд ли понадобться
\begin{description}
\item[Artifacts] 
%\textbf{test} 

Phenomena like contouring, which are not present in the source image but
produced by the digital signal processing, are called artifacts.  The most
common type of artifact is the Moire' pattern.  If you display or print an
image of several lines, closely spaced and radiating from a single point,
you will see what appear to be flower-like patterns.  These are not part of
the original image but are an illusion produced by the jaggedness of the
display.\cite{Dh}
Error Noise

Returning to our fixed-thresholded (and badly-rendered) image, how could we
document what has taken place to make this image so inaccurate?  Expressing
it in technical terms, a relatively large amount of error "noise" is present
in the fixed-thresholded image.  The error value is the difference between
the image's original intensity at a given dot and the intensity of the
displayed dot.  Obviously, very dark values like 1 or 2 (which are almost
full black) incur very small errors when they are rendered as a 0 value
(black) dot.  On the other hand, a gross error is incurred when a 129 value
dot (a medium gray) is displayed at 255 value (white), for instance.\cite{Dh}

Fixed Thresholding

A good place to start is with the example of performing a simple (or fixed)
thresholding operation on our grayscale image in order to display it on our
black and white device.  This is accomplished by establishing a demarcation
point, or threshold, at the 50\% gray level.  Each dot of the source image is
compared against this threshold value: if it is darker than the value, the
device plots it black, and if it's lighter, the device plots it white.

What happens to the image during this operation?  Well, some detail
survives, but our perception of gray levels is completely gone.  This means
that a lot of the image content is obliterated.  Take an area of the image
which is made up of various gray shades in the range of 60-90\%.  After fixed
thresholding, all of those shades (being darker than the 50\% gray threshold)
will be mapped to solid black.  So much for variations of intensity.

Another portion of the image might show an object with an increasing,
diffused shadow across one of its surfaces, with gray shades in the range of
20-70\%.  This gradual variation in intensity will be lost in fixed
thresholding, giving way to two separate areas (one white, one black) and a
distinct, visible boundary between them.  The situation where a transition
from one intensity or shade to another is very conspicuous is known as
contouring.\cite{Dh}
ШУМЫ(нужно тоже дать определение)
\end{description}

%%% Local Variables:
%%% mode: latex
%%% TeX-master: "rpz"
%%% End:
