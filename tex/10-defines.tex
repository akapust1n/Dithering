\Defines % Необходимые определения. Вряд ли понадобться
\begin{description}
	
	%\textbf{test} 
	\item[Артефакт]
	аномалия, возникающая во время визуального представления изображения\cite{Wiki_artifact}
	\item[Цифровой шум] дефект изображения, вносимый фотосенсорами и электроникой устройств, которые их используют вследствие несовершенства технологий.
	Цифровой шум заметен на изображении в виде наложенной маски из пикселей случайного цвета и яркости\cite{Wiki_noise}
	
	\item[Фиксированная пороговая обработка (fixed thresholding)] -обработка пикселя на основе какого-то фиксированного порового значения. В случае если текущее значения пикселя больше этого порогового значения, пиксель закрашивает одним фиксированным цветом, иначе -другим. 
	\cite{Dh}
	\item[Интенсивность цвета]  степень отличия хроматического цвета от равного ему по светлоте ахроматического, «глубина» цвета. Два оттенка одного тона могут различаться степенью блёклости. При уменьшении насыщенности каждый хроматический цвет приближается к серому.\cite{Wiki_intens}
	\item[Ахроматические цвета] оттенки серого (в диапазоне белый — чёрный).\cite{Wiki_achrom}
	\item[Монохромное изображение] изображение, содержащее свет одного цвета (длины волны), воспринимаемый, как один оттенок (в отличие от цветного изображения, содержащего различные цвета).\cite{Wiki_hrom}
	\item[Цвета шума]система терминов, приписывающая некоторым видам стационарных шумовых сигналов определённые цвета исходя из аналогии между спектром сигнала произвольной природы.\cite{Wiki_color_noise}
	\item[Спектральная плотность S(w) стационарного случайного процесса x(t) ] это частотная функция, характеризующая спектральный (частотный) состав процесса, и представляет собой частотную характеристику для средних значений квадратов амплитуд гармоник, на которые может быть разложен случайный процесс.\cite{Wiki_staz}
	\item[Светлый пиксель] пиксель,код цвета которого более или равен 128 для одноцветной палитры 0-255.
	\item[Темный пиксель] пиксель,код цвета которого менее  128 для одноцветной палитры 0-255.
	
	
	
\end{description}

%%% Local Variables:
%%% mode: latex
%%% TeX-master: "rpz"
%%% End:
