\chapter{Аналитический раздел}
\label{cha:analysis}
%
% % В начале раздела  можно напомнить его цель
%
Алгоритмы дизеринга подразделяются на следующие категории:

\begin{itemize}
\item  Случайный  дизеринг(Random dither);
\item Шаблонный дизеринг(Patterning);
\item Упорядоченный дизеринг(Ordered);
\item Дизеринг рассеивания ошибок(Error-diffusion).
\end{itemize}

Каждый следующий метод лучше по качеству предыдущего, однако, такие сообржажение как время, экономия памяти и прочие могут склонить наш выбор в пользу более простого алгоритма.
Для того, чтобы применить любой из первых трех методов в цвете, нужно просто применить алгоритм отдельно для каждого цвета отдельно и смешать полученные значения.
Для простоты описания, будем рассказывать только о черно-белых изображениях.

% Обратите внимание, что включается не ../dia/..., а inc/dia/...
% В Makefile есть соответствующее правило для inc/dia/*.pdf, которое
% берет исходные файлы из ../dia в этом случае.



\section{Случайный дизеринг}
Этот алгоритм тривиален.  По сравнению с остальными алгоритмами, его качество слишком низко, поэтому он применяется лишь там, где необходима  выоская скорость работы в ущерб качеству. 
Для каждой точки в нашем черно-белом изображении мы генерируем случайное число в диапазоне 0-255: если случайное число больше, чем значение в данной точке, то отображаем белую точку, иначе отображаем черную точку.
Это создает изображение с большим количеством «белого шума»,  который выглядит как «снег». Хотя изображение выглядит неточным и зернистым, оно не содержит артефактов. К слову, это метод дизеринга полезен при воспроизведении низкокачественных изображений, где отсутствие артефактов более важно, чем наличие шумов. Например, изображение содержит градиент всех уровней от черного до белого. Это изображение будет отлично выглядеть после того, как к нему применят случайный дизеринг, а остальные методы дизеринга приведут к возникновению артефактов. 

\section{Шаблонный дизеринг}
Шаблонный дизеринг подрузамевает то, что мы увеличиваем разрешение изображения для того, чтобы оно смотрелось более красиво для человеческого глаза. Так же, как и случайный дизеринг, это простой алгоритм, но он гораздо более эффективен.
Для каждой точки изображения мы генерируем «шаблон» пикселей, который аппроксимирует эту точку. Тем  самым мы можем имтировать больший набор оттенков, чем поддерживает наша глубина цвета.
Например, шаблон 3х3. Он имеет 512 вариантов возможных расположений пикселей, но нас интересует только интенсивность цвета. Она формируется на основе количества черных пикселей, содержащихся в шаблоне. То есть возможных вариантов 10. \\
%\begin{verbatim}

$\begin{vmatrix}
\square&\square&\square \\
\square &\square&\square \\
\square &\square&\square 
\end{vmatrix}$
$\begin{vmatrix}
\square&\square&\square \\
\square &\blacksquare&\square \\
\square &\square&\square 
\end{vmatrix}$ 
$\begin{vmatrix}
\square&\square&\square \\
\square &\blacksquare&\blacksquare \\
\square &\square&\square 
\end{vmatrix}$ 
$\begin{vmatrix}
\square&\blacksquare&\square \\
\square &\blacksquare&\blacksquare \\
\square &\square&\square 
\end{vmatrix}$ 
$\begin{vmatrix}
\square&\blacksquare&\blacksquare \\
\square &\blacksquare&\blacksquare \\
\square &\square&\square 
\end{vmatrix}$ 
$\begin{vmatrix}
\square&\blacksquare&\blacksquare \\
\square &\blacksquare&\blacksquare \\
\square &\blacksquare&\square 
\end{vmatrix}$ 
$\begin{vmatrix}
\square&\blacksquare&\blacksquare \\
\blacksquare &\blacksquare&\blacksquare \\
\square &\blacksquare&\square 
\end{vmatrix}$ 

$\begin{vmatrix}
\square&\blacksquare&\blacksquare \\
\blacksquare &\blacksquare&\blacksquare \\
\blacksquare &\blacksquare&\square 
\end{vmatrix}$ 	 
$\begin{vmatrix}
\square&\blacksquare&\blacksquare \\
\blacksquare &\blacksquare&\blacksquare \\
\blacksquare &\blacksquare&\blacksquare 
\end{vmatrix}$ 
$\begin{vmatrix}
\blacksquare&\blacksquare&\blacksquare \\
\blacksquare &\blacksquare&\blacksquare \\
\blacksquare &\blacksquare&\blacksquare 
\end{vmatrix}$ 
%\end{verbatim}


\section{Упорядоченный дизеринг}
%возможно, это вранье, ПЕРЕЧИТАТЬ ОРИГИНАЛ ЕЩЁ РАЗ
Значительны недостатком шаблонного дизеринга явлется пространственное увеличение картинки(и увеличение её разрешения). Упорядоченный дизеринг позволяет избежать этого пространственного искажения. Очевидно, что для того, чтобы достичь этого, каждая точка в исходном изображении должны быть сопоставляется c пикселем на конечном изображении  один-к-одному.Существуют два вида упорядоченного дизеринга: кластерный и дисперсный.
Суть этих методов заключается в том, что мы разбиваем исходное изображение на квадраты пикселей и значения маски в каждой точке квадрата выступает в роли порогового значения. Если значение пикселя(отмастшабированное под интервал маски) в данной точке больше значения маски, то красим пиксель в черный цвет, иначе в белый. Кластерный паттерны выглядят вот так:\\
$\begin{vmatrix}
8&  3 & 4  \\                      
6&  1&  2   \\                 
7 & 5 & 9  \\
\end{vmatrix}    $   ~    ~  ~  ~        
$\begin{vmatrix}
1 &  7 & 4 \\
5 & 8 & 3 \\
6 & 2 & 9 \\
\end{vmatrix}$

Кластерные паттерны применяются в случаях когда понятие «конкретный пиксель» у устройства вывода информации отсутсвует(ЭЛТ-мониторыи подобное). Во многих исследованиях[ссылка на Байера] было отмечено, что если устройство вывода позволяет применить дисперсный метод, то его применение является предпочтительным.Так же Байер \cite{Bayer} показал, что для матриц порядков степени двух существует оптимальная структура дисперсных точек, которая приводит к наименьшему количеству шумов(для матрицы 2х2 и 4х4 соотвественно):\\
$\begin{vmatrix}
1 & 3 \\          
4 & 2   \\
\end{vmatrix}$
 \\
 $\begin{vmatrix}
1& 9& 3& 11\\
13& 5& 15& 7\\
4& 12& 2& 10\\
16& 8&14&6\\
\end{vmatrix}$
В конструкторской части показан рекурсивный метод для создания масок бОльшего размера. Основным недостатком данного метода считается то, что в результате его работы формируется большое количество артефактов.
\section{Дизеринг при помощи диффузии ошибок}
Наиболее качественным методом является метод рассеивания ошибок. Но так же он, к сожалению, самый медленный. Существуют несколько вариантов этого алгоритма, причем скорость алгоритма обратно пропорционально качеству изображения.
Суть алгоритма: для каждой точки изображения находим ближайший возможный цвет. Затем мы рассчитываем разницу между текущим значеним и ближайшим возможным. Эта разница и будем нашем значением ошибки.Это значение ошибки мы распределяем между соседними элементами, которые мы ещё не посещали. Для последних точек ошибка распределяется между уже посещенными точками.
\section{Вариации алгоритма дизернга при помощи диффузии ошибок}
Линия сканирование движется слева-направо. Когда линия сканирования доходит до конца горизонтальной строки пикселей, переходим с первому пикселю следующий строки и повторяем необходимые действия.\\
 \textit{Примечание: числа на схемах - это доли от значения  ошибки. Например, 7/16 на схеме выглядит как 7. То есть 7 обозначает некую величину, равную значение ошибки*7/16 }
  \subsection{Фильтр(маска??) Флойда-Cтейнберга }
  Каждый пиксель распределяет свою ошибку Флойд и Стейнберг тщательно выбрал этот подбирали коэффицианты таким образом, что в районах с  интенсивностью 1/2 от общего количество оттенков, изображение выглядело похожим на шахматную доску.\\
  P ~* ~7\\
  3 ~5 ~1~ (1/16)
  \subsection{"Ложный"  фильтр Флойда-Стейнберга }
  В случае сканирования слева-направо этот фильтр порождает большое количество артефактов.Чтобы получить изображение с мЕньшим количеством артефактов, нужно четные строки сканировать справа-налево, а нечетные строки сканировать слева-направо.\\
  *~ 3\\
  3~ 2~ (1/8)
  \subsection{Фильтр Джарвиса,Джунка и Нинка}
    В случае когда фильтры Флойда-Стейдберга дают недостаточно хороший результат, применяют фильтры с более широким распределением ошибки. Фильтр Джарвиса, Джунка и Нинка требует связи с 12 соседями, что очевидно ведет в большим затратам памяти и времени:\\
        P~   P~   *~   7~   5 \\
        3~   5~   7~   5~   3\\
        1~   3~   5~   3~   1 ~   (1/48)
  \subsection{Фильтр Стаки}
  Стаки переработал фильтр Джарвиса, Джунка и Нинка.После такого как мы вычислим 8/42 ошибки, остальные значения можно получить при помощи побитовых сдвигов, тем самым сокращая время работы алгоритма.\\
     P~   P~  *~   8~   4\\
     2~   4~   8~   4~   2\\
     1~  2~   4~   2~   1 ~   (1/42)
  \subsection{Фильтр Бурка}
  Бурк переработал фильтр Стаки. Результат можно получить чуть быстрее за счет использования побитовых операций.\\
           P~ P~ *~   8~   4~  \\           
           2~   4~   8~   4~   2~   (1/32)
        
  
%%% Local Variables:
%%% mode: latex
%%% TeX-master: "rpz"
%%% End:
