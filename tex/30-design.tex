\chapter{Конструкторский раздел}
\label{cha:design}
\section{Оценка качества изображений}
\subsection{PSNR}
Пиковое отношение сигнала к шуму (англ. peak signal-to-noise ratio) - соотношение между максимумом возможного значения сигнала и мощностью шума, искажающего значения сигнала.\cite{Habr1} 
\begin{equation}
PSNR=20\log_{10} \frac{MAX_i}{\sqrt{MSE}} 
\label{F:F1}
\end{equation}
Где $MAX_i$ - это максимальное значение, принимаемое пикселем изображения, MSE - среднеквадратичное отклонение.  Для двух монохромных изображений I и K размера m×n, одно из которых считается зашумленным приближением другого, вычисляется так:
\begin{equation}
MSE =\frac{1}{m*n}\sum_{i=0}^{m-1}\sum_{j=0}^{n-1}|I(i,j)-K(i,j)|^2
\label{F:F2}
\end{equation}
\subsection{SSIM}
Индекс структурного сходства (SSIM от англ. structure similarity) — метод измерения схожести между двумя изображениями путем полного сопоставления. SSIM-индекс является развитием традиционных методов, таких как PSNR (peak signal-to-noise ratio) и метод среднеквадратичной ошибки MSE, которые оказались несовместимы с физиологией человеческого восприятия.

Отличительной особенностью метода, в отличие от MSE и PSNR, является то, что он учитывает «восприятие ошибки» благодаря учёту структурного изменения информации. Идея заключается в том, что пиксели имеют сильную взаимосвязь, особенно когда они близки пространственно. Данные зависимости несут важную информацию о структуре объектов и о сцене в целом.
 Особенностью является, что SSIM всегда лежит в промежутке от -1 до 1, причем при его значении равном 1, означает, что мы имеем две одинаковые картинки. Общая формула имеет вид
\begin{equation}
SSIM(x,y) = \frac{(2\mu_x\mu_y +c_1)(\sigma_xy+c_2)}{(\mu^2_x+\mu^2_y+c_1)(\sigma^2_x+\sigma^2_y+c_2)}
\label{F:F3}
\end{equation}
Тут $\mu_x$ среднее значение для первой картинки, $\mu_y$  для второй, $\sigma_x$ среднеквадратичное отклонение для первой картинки, и соотвественно $\sigma_y$ для второй, $\sigma_xy$ это уже ковариация. Она находится следующим образом:
\begin{equation}
\sigma_xy = \mu_xy - \mu_x\mu_y
\label{F:F4}
\end{equation} 
%я не понимаю что это за формула, пишу как читаю
  $c_1$ и $c_2$ -  поправочные коэффициенты, которые нужны вследствие малости знаменателя.
\begin{equation}
c_1= (0,01*d)^2  
\end{equation}
\begin{equation}
c_2=(0,03*d)^2
\label{F:F5}
\end{equation}  
  d - количество цветов, соответствующих данной битности изображения 
  
%%% Local Variables:
%%% mode: latex
%%% TeX-master: "rpz"
%%% End:
