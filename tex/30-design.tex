\chapter{Конструкторский раздел}
\label{cha:design}
\section{Оценка качества изображений}
\subsection{PSNR}
Пиковое отношение сигнала к шуму (англ. peak signal-to-noise ratio) - соотношение между максимумом возможного значения сигнала и мощностью шума, искажающего значения сигнала.\cite{Habr1} 
\begin{equation}
PSNR=20\log_{10} \frac{MAX_i}{\sqrt{MSE}} 
\label{F:F1}
\end{equation}
Где $MAX_i$ - это максимальное значение, принимаемое пикселем изображения, MSE - среднеквадратичное отклонение.  Для двух монохромных изображений I и K размера m×n, одно из которых считается зашумленным приближением другого, вычисляется так:
\begin{equation}
MSE =\frac{1}{m*n}\sum_{i=0}^{m-1}\sum_{j=0}^{n-1}|I(i,j)-K(i,j)|^2
\label{F:F2}
\end{equation}
\subsection{SSIM}
Индекс структурного сходства (SSIM от англ. structure similarity) — метод измерения схожести между двумя изображениями путем полного сопоставления. SSIM-индекс является развитием традиционных методов, таких как PSNR (peak signal-to-noise ratio) и метод среднеквадратичной ошибки MSE, которые оказались несовместимы с физиологией человеческого восприятия.

Отличительной особенностью метода, в отличие от MSE и PSNR, является то, что он учитывает «восприятие ошибки» благодаря учёту структурного изменения информации. Идея заключается в том, что пиксели имеют сильную взаимосвязь, особенно когда они близки пространственно. Данные зависимости несут важную информацию о структуре объектов и о сцене в целом.
 Особенностью является, что SSIM всегда лежит в промежутке от -1 до 1, причем при его значении равном 1, означает, что мы имеем две одинаковые картинки. Общая формула имеет вид
\begin{equation}
SSIM(x,y) = \frac{(2\mu_x\mu_y +c_1)(\sigma_xy+c_2)}{(\mu^2_x+\mu^2_y+c_1)(\sigma^2_x+\sigma^2_y+c_2)}
\label{F:F3}
\end{equation}
Тут $\mu_x$ среднее значение для первой картинки, $\mu_y$  для второй, $\sigma_x$ среднеквадратичное отклонение для первой картинки, и соотвественно $\sigma_y$ для второй, $\sigma_xy$ это уже ковариация. Она находится следующим образом:
\begin{equation}
\sigma_xy = \mu_xy - \mu_x\mu_y
\label{F:F4}
\end{equation} 
%я не понимаю что это за формула, пишу как читаю
  $c_1$ и $c_2$ -  поправочные коэффициенты, которые нужны вследствие малости знаменателя.
\begin{equation}
c_1= (0,01*d)^2 
\label{F:F5} 
\end{equation}
\begin{equation}
c_2=(0,03*d)^2
\label{F:F6}
\end{equation}  
  d - количество цветов, соответствующих данной битности изображения 
\section{Виды случайных распределений} 
\subsection{Белый шум}
Белый шумом называют сигнал с равномерной спектральной плотностью на всех частотах и дисперсией, равной бесконечности. Является стационарным случайным процессом.
В качестве сигнала в задаче дизеринга  рассматриватся последовать последовательность чисел, получаемых от генератора случайных чисел.

\begin{figure}[H]
	\centering
    \includegraphics[width=\textwidth]{img/2_white_noise.png}
	\caption{Диаграмма белого шума//почему-то очень много места сверху}
	\label{fig:spire05}
\end{figure}

\subsection{Красный шум}
Спектральная плотность красного шума пропорциональна 1/f², где f — частота. Это означает, что на низких частотах шум имеет больше энергии, чем на высоких.																									\begin{figure}[h]
	\centering
 \includegraphics[width=\textwidth ]{img/3_red_noise.png}
	\caption{Диаграмма красного шума }
	\label{fig:spire04}
\end{figure} 
Коричневый шум можно получить при помощи алгоритма случайного блуждания. Выбиратся наугад число в определенном диапазоне,
%Я НЕ УВЕРЕН, ЧТО ЭТО И ЕСТЬ АЛГОРИТМ СЛУЧАЙНОГ БЛУЖДАНИЯ
  прибавляются к значению предыдущего числа(тоже полученного случайно) и делится пополам.Например, если первое значение равно 103, а нам случайно выпало число 67, то следующее значение будет 170. Если  построить так длинную последовательность значений, получится красный шум. 
  \begin{lstlisting}[style=pseudocode,caption={Получение коричневого шума}] 
  def smoother(noise):
     output = []
     for i in range(len(noise) - 1):
          output.append(0.5 * (noise[i] + noise[i+1]))
  return output
  \end{lstlisting}
 \subsection{Гауссовский шум}
 
 Гауссовский шум  - шум, имеющий функцию плотности вероятности (PDF), равную нормальному распределению, которое также известно как гауссово распределение. 

 \begin{equation}
 p_g(z)=\frac{1}{\sigma\sqrt{2*\pi}}e^{-\frac{(z-\mu)^2}{2\sigma^2}}
 \end{equation}
 z - количество цветов, $\mu$ -среднее значение, $\sigma$ - стандартное отклонение.
  \begin{figure}
  	\centering
  	%  \includegraphics[width=\textwidth]{inc/svg/pic01}
  	\caption{Диаграмма гауссовского шума}
  	\label{fig:spire02}
  \end{figure}
 \subsection{Розовый шум}
 Розовый шум представляет собой что-то среднее между коричневым шумом и белым шумом.Получение его аналогично получению коричневого шума.
 \begin{lstlisting}[style=pseudocode,caption={Получение розового шума}]
 def rougher(noise):
     output = []
     for i in range(len(noise) - 1):
         output.append(0.5 * (noise[i] - noise[i+1]))
 return output
 \end{lstlisting}
 \subsection{Алгоритм Флойда-Стейнберга}
 Рассмотрим более детально алгоритм Флойда-Стейнберга.
\\\\
 \begin{figure}
 	\centering
 	 \includegraphics[width=\textwidth]{img/1.png}
 	\caption{Схема рассеивания ошибки}
 	\label{fig:spire03}
 \end{figure}
 
   \begin{lstlisting}[style=pseudocode,caption={Алгоритм Флойда-Стейнберга}]
   for(x = 0; x < width; x++)
       for(y = 0; y < height; y++){
           P(x,y) = trunc(I(x,y)+0.5);
           e = I(x,y) - P(x,y);
           I(x,y+1) += \alpha*e;
           I(x+1, y-1) += \beta*e;
           I(x+1, y) = \gamma*e;
           I(x+1, y+1) +=\sigma*e; }
   \end{lstlisting}
%%% Local Variables:

%%% mode: latex
%%% TeX-master: "rpz"
%%% End:
