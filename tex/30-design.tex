\chapter{Конструкторский раздел}
\label{cha:design}
\section{Оценка качества изображений}
\subsection{PSNR}
Пиковое отношение сигнала к шуму (англ. peak signal-to-noise ratio) - соотношение между максимумом возможного значения сигнала и мощностью шума, искажающего значения сигнала.\cite{Habr1} 
\begin{equation}
PSNR=20\log_{10} \frac{MAX_i}{\sqrt{MSE}} 
\label{F:F1}
\end{equation}
Где $MAX_i$ - это максимальное значение, принимаемое пикселем изображения, MSE - среднеквадратичное отклонение.  Для двух монохромных изображений I и K размера m×n, одно из которых считается зашумленным приближением другого, вычисляется так:
\begin{equation}
MSE =\frac{1}{m*n}\sum_{i=0}^{m-1}\sum_{j=0}^{n-1}|I(i,j)-K(i,j)|^2
\label{F:F2}
\end{equation}
\subsection{SSIM}
Индекс структурного сходства (SSIM от англ. structure similarity) — метод измерения схожести между двумя изображениями путем полного сопоставления. SSIM-индекс является развитием традиционных методов, таких как PSNR (peak signal-to-noise ratio) и метод среднеквадратичной ошибки MSE, которые оказались несовместимы с физиологией человеческого восприятия.

Отличительной особенностью метода, в отличие от MSE и PSNR, является то, что он учитывает «восприятие ошибки» благодаря учёту структурного изменения информации. Идея заключается в том, что пиксели имеют сильную взаимосвязь, особенно когда они близки пространственно. Данные зависимости несут важную информацию о структуре объектов и о сцене в целом.
Особенностью является, что SSIM всегда лежит в промежутке от -1 до 1, причем при его значении равном 1, означает, что мы имеем две одинаковые картинки. Общая формула имеет вид
\begin{equation}
SSIM(x,y) = \frac{(2\mu_x\mu_y +c_1)(\sigma_xy+c_2)}{(\mu^2_x+\mu^2_y+c_1)(\sigma^2_x+\sigma^2_y+c_2)}
\label{F:F3}
\end{equation}
Тут $\mu_x$ среднее значение для первой картинки, $\mu_y$  для второй, $\sigma_x$ среднеквадратичное отклонение для первой картинки, и соотвественно $\sigma_y$ для второй, $\sigma_xy$ это уже ковариация. Она находится следующим образом:
\begin{equation}
\sigma_xy = \mu_xy - \mu_x\mu_y
\label{F:F4}
\end{equation} 
%я не понимаю что это за формула, пишу как читаю
$c_1$ и $c_2$ -  поправочные коэффициенты, которые нужны вследствие малости знаменателя.
\begin{equation}
c_1= (0,01*d)^2 
\label{F:F5} 
\end{equation}
\begin{equation}
c_2=(0,03*d)^2
\label{F:F6}
\end{equation}  
d - количество цветов, соответствующих данной битности изображения 
Для подтверждения или опровержения вышеописанных гипотезы реализуются несколько алгоритмов случайного распределения, алгоритм Флойда-Стейнберга, модификации на основе алгоритма Флойда-Стейнберга. Результаты работы алгоритмов сравниваются по времени работы, по количеству затрачиваемой памяти,а так же по SSIM и PSNR.
\subsection{Алгоритм случайного распределения} 
P(x,y)  - цвет конкретного пикселя
\begin{lstlisting}[style=pseudocode,caption={Алгоритм случайного распределения}] 
for x in range(height):
    for y in range(weight):
        if P(x,y)>127:
            P(x,y) = 255
        else:
            P(x,y) = 0
\end{lstlisting}

\section{Виды случайных распределений} 
\subsection{Белый шум}
Белый шумом называют сигнал с равномерной спектральной плотностью на всех частотах и дисперсией, равной бесконечности. Является стационарным случайным процессом.
В качестве сигнала в задаче дизеринга  рассматриватся последовать последовательность чисел, получаемых от генератора случайных чисел.
\\\\\\\\\\\\\\\\\\\\\\\\\\\
\begin{figure}[h!]
	\centering
	\includegraphics[width=\textwidth]{img/8_white_noise.png}
	\caption{Диаграмма белого шума}
	\label{fig:spire05}
\end{figure}

\subsection{Коричневый шум}
Спектральная плотность коричневого шума пропорциональна 1/f², где f — частота.																									
Это означает, что на низких частотах шум имеет больше энергии, чем на высоких. То есть пикселей темных цветов большей, чем пикселей светлого цвета.Применение фильтра коричневого шума в целом затемняет получаемое изображение.
\\\
\begin{figure}[h]
	\centering
	\includegraphics[width=\textwidth ]{img/6_brown_noise.png}
	\caption{Диаграмма красного шума }
	\label{fig:spire04}
\end{figure} 
\begin{lstlisting}[style=pseudocode,caption={Получение коричневого шума}] 
def smoother(noise):
    output = []
    for i in range(len(noise) - 1):
        output.append(0.5 * (noise[i] + noise[i+1]))
return output
\end{lstlisting}

\subsection{Гауссовский шум}

Гауссовский шум  - шум, имеющий функцию плотности вероятности (PDF), равную нормальному распределению, которое также известно как гауссово распределение. 

\begin{equation}
p_g(z)=\frac{1}{\sigma\sqrt{2*\pi}}e^{-\frac{(z-\mu)^2}{2\sigma^2}}
\end{equation}
z - количество цветов, $\mu$ -среднее значение, $\sigma$ - стандартное отклонение.
%\begin{figure}
%	\centering
%  \includegraphics[width=\textwidth]{inc/svg/pic01}
%	\caption{Диаграмма гауссовского шума}
%	\label{fig:spire02}
% \end{figure}
\subsection{Фиолетовый шум}
Фиолетовый  шум представляет собой противооложноть между коричневому шуму. Получение его аналогично получению коричневого шума.Применение фильтра фиолетового шума в целом засветляет получаемое изображение.
\\\\\\\\
\begin{figure}[h!]
	\centering
	\includegraphics[width=\textwidth]{img/7_pink_noise.png}
	\caption{Диаграмма розового шума}
	\label{fig:spire02}
\end{figure}\\
\begin{lstlisting}[style=pseudocode,caption={Получение розового шума}]
def rougher(noise):
    output = []
    for i in range(len(noise) - 1):
    output.append(0.5 * (noise[i] - noise[i+1]))
return output
\end{lstlisting}

\subsection{Розовый и синий шумы}
Розовый и синий шумы представляют собой "промежуточные" шумы.Изобржаение с розовым шумом темнее  изображения с белым шумом, но светлее изображения с коричневым шумом. Изображение с синим шумом светлее изображения с белым шумом, но темнее чем изображение с фиолетовым шумом. Их получение аналогично получению со коричневого и фиолетового шумов.
\\
\\
\\

\section{Алгоритм Байера}
  
  \begin{lstlisting}[style=pseudocode,caption={Алгоритм Флойда-Стейнберга}]
For each pixel, Input,:
    Factor  = ThresholdMatrix[xcoordinate % X][ycoordinate % Y];
    Attempt = Input + Factor * Threshold
    Color = FindClosestColorFrom(Palette, Attempt)
  \end{lstlisting}
\section{Алгоритм Флойда-Стейнберга}
Рассмотрим более детально алгоритм Флойда-Стейнберга.
P(x,y) - цвет пикселя в точке x,y
I(x,y) - предполгаемый цвет пикселя с учетом ошибки(действительное число)
\begin{figure}[h!]
	\centering
 	\includegraphics[width=\textwidth]{img/1.png}
	\caption{Схема рассеивания ошибки}
	\label{fig:spire03}
\end{figure}
Следует отметить, что сумма "долей" рассевания ошибки и для других алгоритмов упорядоченного дизеринга будет равна единице.
\begin{lstlisting}[style=pseudocode,caption={Алгоритм Флойда-Стейнберга}]
for x in range(width):
    for y in range(height):
       P(x,y) = trunc(I(x,y)+0.5)
       e = I(x,y) - P(x,y)
       I(x,y-1) += 7/16*e
       I(x+1, y-1) += 3/16*e
       I(x+1, y) += 5/16*e
       I(x+1, y+1) += 1/16*e
\end{lstlisting}
\section{Ложный алгоритм Флойда-Стейнберга}
Отличительной особенностью ложного алгоритма Флойда-Стейнберга является по пикселям изображения справа-налево, а не слева-направо как в большинстве других алгоритмов.
\begin{lstlisting}[style=pseudocode,caption={Ложный алгоритм Флойда-Стейнберга}]
for x in range(width):
    for y in range(height):
        P(x,y) = trunc(I(x,y)+0.5)
        e = I(x,y) - P(x,y)
        P(x,y) = trunc(I(x,y)+0.5)
        e = I(x,y) - P(x,y)
        I(x,y-1) += 3/8*e
        I(x-1, y-1) += 2/8*e
        I(x-1, y) += 3/8*e
\end{lstlisting}
\section{Алгоритм Джарвиса,Джунка и Нинка}

\begin{lstlisting}[style=pseudocode,caption={Алгоритм Джарвиса,Джунка и Нинка}]
for x in range(width):
    for y in range(height):
        P(x,y) = trunc(I(x,y)+0.5)
        e = I(x,y) - P(x,y)
        P(x,y) = trunc(I(x,y)+0.5)
        e = I(x,y) - P(x,y)
        I(x,y-1) += 3/48*e
        I(x+1,y+1) += 5/48*e
        I(x,y+1) += 7/48*e
        I(x-1,y+1) += 5/48*e
        I(x-2,y+1) += 3/48*e
        I(x-1,y+1) += 5/48*e
        I(x+2,y+2) += 1/48*e
        I(x+1,y+2) += 3/48*e
        I(x, y+2) += 5/48*e
        I(x-1,y+2) += 3/48*e
        I(x-2,y+2) += 1/48*e
        I(x+1,y) += 7/48*e
        I(x+2,y) += 5/48*e

\end{lstlisting}


\bigskip

\section{Первый алгоритм Юлиомы}
\begin{lstlisting}[style=pseudocode,caption={Первый алгоритм Юлиомы}]
For each pixel, Input, in the original picture:
    Factor  = ThresholdMatrix[xcoordinate % X][ycoordinate % Y];
    Make a Plan, based on Input and the Palette.

    If Factor < Plan.ratio,
        Draw pixel using Plan.color2
    else,
        Draw pixel using Plan.color1
\end{lstlisting}

\begin{lstlisting}[style=pseudocode,caption={План нахождения цвета пикселя}]
SmallestPenalty = 10^99 
For each unique combination of two colors from the palette, Color1 and Color2:
    For each possible Ratio, 0 .. (X*Y-1):
        Mixed = Color1 + Ratio * (Color2 - Color1) / (X*Y)
        Penalty = Evaluate the difference of Input and Mixed.

        If Penalty < SmallestPenalty,
        SmallestPenalty = Penalty
        Plan = { Color1, Color2, Ratio / (X*Y) }1
\end{lstlisting}
\begin{lstlisting}[style=pseudocode,caption={Вспомогательная матрица перого Алгоритма Юлиомы}]
     [0/64, 48/64, 12/64, 60/64, 3/64, 51/64, 15/64, 63/64,
     32/64, 16/64, 44/64, 28/64, 35/64, 19/64, 47/64, 31/64,
     8/64, 56/64, 4/64, 52/64, 11/64, 59/64, 7/64, 55/64,
     40/64, 24/64, 36/64, 20/64, 43/64, 27/64, 39/64, 23/64,
     2/64, 50/64, 14/64, 62/64, 1/64, 49/64, 13/64, 61/64,
     34/64, 18/64, 46/64, 30/64, 33/64, 17/64, 54/64, 29/64,
     10/64, 58/64, 6/64, 54/64, 9/64, 57/64, 5/64, 53/64,
     42/64, 26/64, 38/64, 22/64, 41/64, 25/64, 37/64, 21/64]
\end{lstlisting}


%%% Local Variables:

%%% mode: latex
%%% TeX-master: "rpz"
%%% End:
%--количество цветов
%||количество пикселей