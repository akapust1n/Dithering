\Introduction

Изображения, хранимые в цифровом виде, представляются как массив из значений атрибутов; при этом для представления полноцветных фотографий используется диапазон из несольких миллионов значений на каждый атрибут. Но  количество выводимых отображающим устройством оттенков ограничено. Если графическое устройство не способно воссоздавать достаточное количество цветов, тогда используют растрирование — независимо от того, растровое это устройство или нерастровое.Такие способы используют особенности человеческого зрения и в первую очередь — пространственную интеграцию. Если достаточно близко расположить маленькие точки разных цветов, то они будут восприниматься как одна точка с некоторым усредненным цветом. Если на плоскости густо расположить много маленьких разноцветных точек, то будет создана визуальная иллюзия закрашивания плоскости определенным усредненным цветом. Однако, если увеличивать размеры точек и (или) расстояние между ними, то иллюзия сплошного закрашивания исчезает — включается другая система человеческого зрения, которая обеспечивает способность различать объекты, подчеркивать контуры.
В компьютерных графических системах часто используют эти методы. Они позволяют увеличить количество оттенков цветов за счет снижения пространственного разрешения растрового изображения(иначе говоря — это обмен разрешающей способности на количество цветов). В компьютерной графике такие методы растрирования получили название дизеринг(eng.dithering - разрежение, дрожание).
В последнее время, в связи с распространиением высокачественного видео(4K видео, $360^{\circ}$ видео), высокачественных фотографий,  проблема сохранения качества изображения при уменьшении его размера  становится все более актуальной.

%\begin{itemize}
%\item проанализировать существующую всячину;
%\item спроектировать свою, новую всячину;
%\item изготовить всякую всячину;
%\item проверить её работоспособность.
%\end{itemize}
