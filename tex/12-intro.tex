\Introduction

Обычно изображения, хранимые в цифровом виде, представляются как массив из значений атрибутов; при этом для представления полноцветных фотографий используется диапазон из несольких миллионов значений на каждый атрибут. Но часто количество выводимых отображающим устройством оттенков ограничено. Если графическое устройство не способно воссоздавать достаточное количество цветов, тогда используют растрирование — независимо от того, растровое это устройство или нерастровое. В полиграфии растрирование известно давно. Оно использовалось несколько столетий тому назад для печати гравюр. В гравюрах изображение создается многими штрихами, причем полутоновые градации представляются или штрихами разной толщины на одинаковом расстоянии, или штрихами одинаковой толщины с переменной густотой расположения. Такие способы используют особенности человеческого зрения и в первую очередь — пространственную интеграцию. Если достаточно близко расположить маленькие точки разных цветов, то они будут восприниматься как одна точка с некоторым усредненным цветом. Если на плоскости густо расположить много маленьких разноцветных точек, то будет создана визуальная иллюзия закрашивания плоскости определенным усредненным цветом. Однако, если увеличивать размеры точек и (или) расстояние между ними, то иллюзия сплошного закрашивания исчезает — включается другая система человеческого зрения, которая обеспечивает способность различать объекты, подчеркивать контуры.
В компьютерных графических системах часто используют эти методы. Они позволяют увеличить количество оттенков цветов за счет снижения пространственного разрешения растрового изображения(иначе говоря — это обмен разрешающей способности на количество цветов) или подмешивание в исходное изображение случайного шума. В литературе по КГ такие методы растрирования получили название dithering (разрежение, дрожание). 

%\begin{itemize}
%\item проанализировать существующую всячину;
%\item спроектировать свою, новую всячину;
%\item изготовить всякую всячину;
%\item проверить её работоспособность.
%\end{itemize}
